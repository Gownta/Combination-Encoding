\documentclass[preprint]{sigplanconf}

\usepackage{graphicx}
\usepackage{amsmath}
\usepackage{amsfonts}
\usepackage{amsthm}
\usepackage{multirow}
\usepackage{listings}
\usepackage{algorithm}
\usepackage{algorithmic}
\usepackage{float}

%\newtheorem{theorem}{Theorem}[section]
%\newtheorem{lemma}[theorem]{Lemma}
%\newtheorem{proposition}[theorem]{Proposition}
%\newtheorem{corollary}[theorem]{Corollary}
%\renewenvironment{proof}[1][Proof]{\begin{trivlist}
%\item[\hskip \labelsep {\bfseries #1}]}{\end{trivlist}}
%\newenvironment{definition}[1][Definition]{\begin{trivlist}
%\item[\hskip \labelsep {\bfseries #1}]}{\end{trivlist}}
%\newenvironment{assumption}[1][Assumption]{\begin{trivlist}
%\item[\hskip \labelsep {\bfseries #1}]}{\end{trivlist}}
%\newenvironment{example}[1][Example]{\begin{trivlist}
%\item[\hskip \labelsep {\bfseries #1}]}{\end{trivlist}}
%\newenvironment{remark}[1][Remark]{\begin{trivlist}
%\item[\hskip \labelsep {\bfseries #1}]}{\end{trivlist}}
%\renewcommand{\qed}{\nobreak \ifvmode \relax \else
%      \ifdim\lastskip<1.5em \hskip-\lastskip
%      \hskip1.5em plus0em minus0.5em \fi \nobreak
%      \vrule height0.75em width0.5em depth0.25em\fi}

\lstset{numbers=left,numberstyle=\tiny,numbersep=3pt,tabsize=2,basicstyle=\ttfamily,showstringspaces=false,language=Python}


\begin{document}

	\title{Combination Encoding}
	\authorinfo{Nicholas Ormrod}
	           {University of Waterloo}
	           {njormrod@uwaterloo.ca}
	\maketitle
	
	%\input{...}

	%%%%%%%%%%%%%%%%%%%%%%%%%%%%%%%%%%%%%%	
	\section{Problem}
	
	For a given $n$ and $k$, there are $\binom{n}{k}$ ways to choose $k$ of $n$ elements.
	To space-efficiently represent an arbitrary choice of elements, it is desirable to encode choices as numbers between $0$ and $\binom{n}{k}-1$.
	The encoding should be efficiently computable and decodable.
	
	... (motivation, abstract)
	
	The reverse lexicographical encoding has several nice properties and allows for efficient computation.

	%This analysis is not concerned with the actual values being chosen.
	%They will be represented hereon as $\{ a_1, a_2, \ldots, a_n \}$.
		
	\begin{figure}[!ht]
		\begin{center}	
		\begin{tabular}{c|ccc}
			& Encoding & & Choice \\
			\cline{2-4}
			\multirow{3}{*}{$\binom{4}{2}$}
			& 5 & $\equiv$ & $\{a_1, a_2\}$ \\
			& 4 & $\equiv$ & $\{a_1, a_3\}$ \\
			& 3 & $\equiv$ & $\{a_1, a_4\}$ \\
			$a_1, a_2,$ 
			& 2 & $\equiv$ & $\{a_2, a_3\}$ \\
			$a_3, a_4$
			& 1 & $\equiv$ & $\{a_2, a_4\}$ \\
			& 0 & $\equiv$ & $\{a_3, a_4\}$ \\
		\end{tabular}
		\caption{Reverse lexicographical encoding for two elements chosen from $a_1,a_2,a_3,a_4$}
		\label{Ex}
		\end{center}
	\end{figure}
	
	
	%%%%%%%%%%%%%%%%%%%%%%%%%%%%%%%%%%%%%%
	\subsection{Pascal's Diagonal}
	
	In the reverse lexicographical encoding, how many of the $\binom{n}{k}$ choices start with the first element, $a_1$?
	Quite simply, if $a_1$ is selected, then the other $k-1$ selections must be made from the remaining $n-1$ elements.
	This can be done in $\binom{n-1}{k-1}$ ways.
	
	More generally, how many of the $\binom{n}{k}$ choices start with the $i$'th element?
	The lowest selected element is $a_i$, so the other $k-1$ selections must be made from among the $n-i$ greater elements.
	This can be done in $\binom{n-i}{k-1}$ ways.
	
	Since the first selected element is between $a_1$ and $a_{n-k+1}$, the following equation must be true:
	
	\begin{eqnarray}
		\binom{n}{k} & = & \sum_{i=1}^{n-k+1}{\binom{n-i}{k-1}}
		\label{CE}
	\end{eqnarray}
	
	Visually, on Pascal's triangle, the underlined cell is the sum of the bolded cells:	
	
	\begin{tabular}{ccccccccccc}
	  & & & & & 1 & & & & & \\
	  & & & & 1 & & 1 & & & & \\
	  & & & 1 & & 2 & & \textbf{1} & & & \\
	  & & 1 & & 3 & & \textbf{3} & & 1 & & \\
		& 1 & & 4 & & \textbf{6} & & 4 & & 1 & \\
	  1 & & 5 & & 10 & & \textbf{\underline{10}} & & 5 & & 1
	\end{tabular}
	\vspace{3mm}	
	
	Mathematically, this property follows by Pascal's rule and induction.
	
	Corollary: In Pascal's triangle, the cummulative sum of the $i$'th diagonal is represented by the $(i+1)$'th diagonal.
	
	
	%%%%%%%%%%%%%%%%%%%%%%%%%%%%%%%%%%%%%%
	\section{Encoding}
	
	Since the encoding is lexicographic, the encodings are grouped foremost by their first selection.
	Hence a choice whose first selection is $a_i$ has a larger encoding than all choices whose first selection is $a_{j>i}$.
	
	We may use equation \ref{CE} to determine the number of choices whose first element is greater than $a_i$:
	
	\begin{eqnarray*}
		&   & \text{\{\# of choices whose first selection is greater than $a_i$\}} \\ 
		& = & \sum_{j=i+1}^{n-k+1}{\text{\{\# of choices whose first selection is $a_j$\}}} \\
		& = & \sum_{j=i+1}^{n-k+1}{\binom{n-j}{k-1}} \\
		& = & \sum_{j=1}^{n-k+1-i}{\binom{n-(j+i)}{k-1}} \\
		& = & \binom{n-i}{k}
	\end{eqnarray*}
	
	Therefore the set of all choices that start with $a_i$ have encodings starting at $\binom{n-i}{k}$.
		
	Within the group of choices that start with $a_i$, the relative ordering of two different choices can be determined by the lexicographic ordering of their tails.
	That is to say, the choice $\{ a_i, a_{v_2}, \ldots, a_{v_{k}} \}$ preceeds $\{ a_i, a_{w_2}, \ldots, a_{w_{k}} \}$ if and only if
	$\{ a_{v_2}, \ldots, a_{v_{k}} \}$ preceeds $\{ a_{w_2}, \ldots, a_{w_{k}} \}$.
	
	Due to the reversing of the lexicographical ordering, the set of all possible tails is the same as the set of all $(k-1)$-choices whose encodings are between $0$ and $\binom{n-i}{k-1}-1$.
	This means we may use recursion to compute the offset within the group.
	
	\begin{algorithm}[!ht] % H for exactly here
	\caption{Recursive Encode}
	\label{alg:E}
	
	\textbf{def $E(n,k,\{a_{v_1},a_{v_2},\ldots,a_{v_k}\})$}
	\begin{algorithmic}[1]
	\IF {$k = 0$}
	\RETURN $0$
	\ELSE
	\STATE $groupstart \gets \binom{n-v_1}{k}$
	\STATE $offset \gets E(n,k-1,\{a_{v_2},\ldots,a_{v_k}\})$
	\RETURN $groupstart + offset$
	\ENDIF
	\end{algorithmic}
	
	%\vspace{0.5cm}
	\end{algorithm}
	\begin{algorithm}[!ht] % H for exactly here
	\caption{Recursive Decode}
	\label{alg:D}
	
	\textbf{def $D(n,k,e)$}
	\begin{algorithmic}[1]
	\IF {$k = 0$}
	\RETURN $[]$
	\ELSE
	\STATE $v \gets \min \{ i : \binom{n-i}{k} \le e \}$
	\STATE $offset \gets e - \binom{n-v}{k}$
	\RETURN $a_v : D(n, k-1, offset)$
	\ENDIF
	\end{algorithmic}
		
	\end{algorithm}
	
	In practice, one simple yet effective optimization is to change the base case to $k=1$.
	When $k=1$, the encoding $e$ simply corresponds to the choice $\{n-e\}$.
	
		
	%%%%%%%%%%%%%%%%%%%%%%%%%%%%%%%%%%%%%%
	\subsection{Modifications}
	
	In the reverse lexicographical encoding, a lower encoding corresponds to a greater lexeme.
	This means that the lowest encoding always corresponds to the greatest lexeme: $\{ a_{n-k+1}, \ldots, a_n \}$.
	
	In general, if encoding $e$ corresponds to $\{ a_{v_1}, \ldots, a_{v_k} \}$ with respect to $n$ and $k$, then it corresponds to $\{ a_{v_1+j}, \ldots, a_{v_k+j} \}$ with respect to $(n+j)$ and $k$.
	Thus, to encode all possible $k$-choices of the $n$ elements $\{ a_{j+1}, a_{j+2}, \ldots, a_{j+n} \}$, we may use the same encoding and decoding function by replacing $n$ with $(n+j)$.
	This is particularly useful to get a zero-indexed implementation to work with one-indexed data.
	
	If encoding $e$ corresponds to $\{ a_{v_1}, \ldots, a_{v_k} \}$ with respect to $n$ and $k$, where each $v_i$ is at least $j+1$,
		then $e$ corresponds to $\{ a_{v_1-j}, \ldots, a_{v_k-j} \}$ with respect to $(n-j)$ and $k$.
	This subtractive encoding can be used to work with deltas.
	That is to say, the choice $\{ a_{v_1}, a_{v_2}, \ldots, a_{v_k} \}$ is represented by $\{ (v_1-0), (v_2-v_1),\ldots,(v_n-v_{n-1}) \}$.
	This is particularly useful when dealing with compositions of fixed size, such as the representation of dice rolls.
	
	\begin{figure}[!ht]
		\begin{center}
		\begin{tabular}{ccccc}
			Encoding & & Choice & & Deltas \\
			\hline
			5 & $\equiv$ & $\{a_1, a_2\}$ & $\equiv$ & $\{1,1\}$ \\
			4 & $\equiv$ & $\{a_1, a_3\}$ & $\equiv$ & $\{1,2\}$ \\
			3 & $\equiv$ & $\{a_1, a_4\}$ & $\equiv$ & $\{1,3\}$ \\
			2 & $\equiv$ & $\{a_2, a_3\}$ & $\equiv$ & $\{2,1\}$ \\
			1 & $\equiv$ & $\{a_2, a_4\}$ & $\equiv$ & $\{2,2\}$ \\
			0 & $\equiv$ & $\{a_3, a_4\}$ & $\equiv$ & $\{3,1\}$ \\
		\end{tabular}
		\caption{Delta Representation}
		\label{DEx}
		\end{center}
	\end{figure}
	
	\begin{algorithm}[!ht]
	\caption{Recursive Encode Deltas}
	\label{alg:Ed}
	
	\textbf{def $E_\delta(n,k,\{\delta_1,\delta_2,\ldots,\delta_k\})$}
	\begin{algorithmic}[1]
	\IF {$k = 0$}
	\RETURN $0$
	\ELSE
	\STATE $groupstart \gets \binom{n-\delta_1}{k}$
	\STATE $offset \gets E_\delta(n-\delta_1,k-1,\{\delta_2,\ldots,\delta_k\})$
	\RETURN $groupstart + offset$
	\ENDIF
	\end{algorithmic}	
	
	\end{algorithm}
	
	
	%%%%%%%%%%%%%%%%%%%%%%%%%%%%%%%%%%%%%%
	\section{Observations}
	
	\begin{enumerate}
		\item Any encoding or decoding algorithm takes $\Omega(k)$ time, since they accept as input or produce as output a list of $k$ values.
		\label{obs:Ok}
		
		\item Any specific $k$-choice of $n$ elements can be represented by the $(n-k)$-choice of elements which were not selected - its dual.
				  We may convert a choice to its dual in $O(n)$ time.
		\label{obs:dual}
		
		\item We may assume that $k \le \frac{n}{2}$.
					If not, then we may convert to/from the dual space in $O(n) \le O(2k) \le \Omega(k)$ time.
		\label{obs:k2n}
		
		\item Computing binomial coefficients from scratch, according to \cite{binom}, takes $O(\min(k,n-k))$ time.
					Since we may assume that $k \le \frac{n}{2}$, this is just $O(k)$.
		\label{obs:Ob}
		
		\item $\log\binom{n}{k} = \Theta(k\log\frac{n}{k})$ when $k \le \frac{n}{2}$.
		\label{obs:Tl}
		
		\item The function $i \mapsto \binom{n-i}{k}$ is monotone.
					Hence finding the minimum $i$ such that $\binom{n-i}{k} \le e$, as in algorithm \ref{alg:D}, may be accomplished with a search algorithm.
		\label{obs:search}
		
	\end{enumerate}
		
	
	%%%%%%%%%%%%%%%%%%%%%%%%%%%%%%%%%%%%%%
	\section{Analysis}
	
	The slowest component of the reverse lexicographical encoding is the need to compute binomial coefficients.
	If the binomial coefficients were pre-computed, then encoding can be done in $O(k)$ time and decoding can be done in $O(k\log\frac{n}{k})$ time.
	Given that the input sizes for these two functions are $k$ and $\log\binom{n}{k}$, respectively, these are excellent runtimes.\footnote{
		Technically, encode should alse be $O(\log\binom{n}{k})$, since it returns an encoding of that many bits.}
	
	An alternate approach exists that takes $O(n)$ time, including time spent computing binomial coefficients.
	
	
%	Caching the binomial coefficients for $n$ and $k$ takes $O(k^2)$ space.
%	This is much less than the $O(\binom{n}{k})$ space required to store a full encoding lookup table, but the spirit of the encoding is to be as space efficient as possible.
%	When the binomial coefficients must be computed on the fly, the regular methods take $O(k^2)$ and $O(k^2\log\frac{n}{k})$ time, respectively.
%	There is, however, an alternate method which takes $O(n)$ time.
	
	Full C++ source code and tests for each method can be found at \textbf{??}.
	
	
	\subsection{Weighted Binary Search}
	
	The encoding algorithm, algorithm \ref{alg:E}, contains $k$ recursive calls.
	Each call computes one binomial coefficient, and does $O(1)$ extra work.
	Thus, if the binomial is computable in $O(1)$, encode is $O(k)$.
	
	The decode algorithm, algorithm \ref{alg:D}, is slightly different.
	It needs to search for the smallest $i$ such that $\binom{n-i}{k} \le e$.
	A regular binary search will take $O(\log n)$ steps; however, a slight modification will produce an amortized $O(\log\frac{n}{k})$.

%	The trick relies on two observations.
%	First, each element is greater than the last, since the choices are sorted.
%	Secondly, the expected difference between two consecutive elements is about $\frac{n}{k}$.
%	
%	Searching between $1$ and $n$ is excessive.
%	Since the decoding is in sorted order, every element is greater than the previous.
%	This means that a lower bound for the $j$'th element, $v_j$, is $v_{j-1}$.  
%	
%	Also, performing a balanced search is not optimal.
%	The lowest of $k$ elements chosen from $1..n$ is very likely to be less than $\frac{n}{2}$.
%	If they were spaced evenly, the first element would be around $\frac{n}{2k}$.
	
	When binary searching for the $j$'th element, $v_j$, make the first partition at $l + \frac{n}{k}$, where $l$, a lower bound for $v_j$, is $v_{j-1}$.
	If the check passes, implying $v_j \le l + \frac{n}{k}$, then $v_j \in \left( l, l + \frac{n}{k} \right]$ - a range of size $\frac{n}{k}$.
	If the check fails, implying $v_j > l + \frac{n}{k}$, then $v_j$ and all future elements are greater than $l + \frac{n}{k}$  - update the lower bound to $l + \frac{n}{k}$ and repeat.
	
	\begin{algorithm}[!ht]
	\caption{Recursive Binary Decode}
	\label{alg:Drb}
	
	\textbf{def $D(n,k,e,l=0)$}
	\begin{algorithmic}[1]
	\IF {$k = 0$}
		\RETURN $[]$
	\ELSIF {$\binom{n-(l+\frac{n}{k})}{k} > e$}
		\RETURN $D(n,k,e,l+\frac{n}{k})$
	\ELSE
		\STATE $v \gets \ldots$ // binary search for $\min\{i:\binom{n-i}{k}\le e\}$ in the range $\left( l, l+\frac{n}{k} \right]$
		\STATE $offset \gets e - \binom{n-v}{k}$
		\RETURN $v : D(n,k-1,offset,v)$
	\ENDIF
	\end{algorithmic}
		
	\end{algorithm}

	In this method, when a partition finally passes, a regular binary search finds the correct value in at most $\log\frac{n}{k}$ further iterations.
	Hence this method performs $O(k\log\frac{n}{k} + \{ \text{\# of fails} \})$ total steps.
	But each failure increases the lower bound by $\frac{n}{k}$.\footnote{
		Or $\frac{n}{k-i}$, which is greater than $\frac{n}{k}$.}
	Since the upper bound is $n$, and the initial lower bound is $0$, there can be at most $k$ failures.
	The total cost is therefore $O(k\log\frac{n}{k})$ to find all $k$ elements.
	
	
	\subsection{Bottom-Up Linear Search}
	
	When the binomial coefficients must be computed on the fly, the regular methods slow down by a factor of $k$.
	In these circumstances, the unlikely hero is linear search.
	
	When using linear search to find the smallest $i$ such that $\binom{n-i}{k} \le e$, the value $\binom{n-i}{k}$ may be computed incrementally in $O(1)$ time, under the identity
	
	\begin{eqnarray*}
		\binom{n}{k} & = & \frac{n}{n-k+1}\cdot\binom{n-1}{k}
	\end{eqnarray*}
	
	The coefficients may even be reused in the recursive calls.
	When the $j$'th element is determined to be $i$, then the $j+1$'th element must be at least $i+1$.
	Hence the first binomial in the $j+1$'th linear search may be computed from the last binomial in the $j$'th search, under the identity
	
	\begin{eqnarray*}
		\binom{n}{k} & = & \frac{k+1}{n+1}\cdot\binom{n+1}{k+1}
	\end{eqnarray*}
	
	The first coefficient of the first search, $\binom{n-1}{k}$, must be computed in $O(k)$ time.
	
	This pattern of reusing the coefficients has the property that, on each update, the numerator decreases by $1$.
	Since $O(1)$ work is done for each coefficient, and there are $n$ valid numerators, decode by linear search takes $O(n+k) = O(n)$ time.
	
	A similar approach may be used to encode sequences.
	
	\begin{algorithm}[!ht]
	\caption{Imperative Linear Decode}
	\label{alg:Dil}
	
	\textbf{def $D(n,k,e)$}
	\begin{algorithmic}[1]
	\STATE $ret \gets []$
	\STATE $j \gets 1$ // lower bound for the next element
	\STATE $c \gets \binom{n-j}{k}$ // invariant: $c = \binom{n-j}{k-i+1}$
	\FOR {$i \gets 1..k-1$}
		\WHILE {$c > e$}
			\STATE $c \gets \frac{c\cdot (n-j-k+i-1)}{n-j}$
			\STATE $j \gets j + 1$
		\ENDWHILE
		\STATE $ret.push\_back(j)$
		\STATE $e \gets e - c$
		\STATE $c \gets \frac{c \cdot (k-i+1)}{n-j}$
		\STATE $j \gets j + 1$
	\ENDFOR
	\STATE $ret.push\_back(n-e)$ // $k=1$ optimization
	\RETURN $ret$
	\end{algorithmic}	
	
	\end{algorithm}	
	

	%%%%%%%%%%%%%%%%%%%%%%%%%%%%%%%%%%%%%%
	\section{Conclusion}
	
	The reverse lexicographical encoding may be used to compactly and efficiently encode and decode specific combinations.
	
	With cached coefficients ($O(k^2)$ space overhead) the encoding runs in $O(k)$ time and decoding runs in $O(k\log\frac{n}{k})$ time.
	In the absence of cached coefficients both encoding and decoding can be run in $O(n)$ time.
	
	The algorithms are short, easy to implement, and easy to test. 
	They may easily be modified to deal with encodings of different indices, or deltas.	
	
	Code may be found at \textbf{??}.
	
	%%%%%%%%%%%%%%%%%%%%%%%%%%%%%%%%%%%%%%
	%% Bibliography
	
	\begin{thebibliography}{9}
		\bibitem{binom}
			Yannis Manolopoulos,
		  \emph{Binomial Coefficient Computation: Recursion or Iteration?}
		  SIGCSE Bulletin,
		  Vol 34, No. 4, 2002 December.
	
		\bibitem{CLRS}
			Cormen, Thomas H., et al.
			\emph{Introduction to Algorithms}.
			3rd ed.
			Cambridge: The MIT Press, 2009.
	\end{thebibliography}
	
	%%%%%%%%%%%%%%%%%%%%%%%%%%%%%%%%%%%%%%
	
	\vspace{1cm}\begin{center}\includegraphics[scale=0.25]{C:/Users/Nicholas/Pictures/Dragons/raptor}\par\end{center}
	
	
%	Interesting math: The expected number of iterations to find the first $i$ is $\frac{n+1}{k+1}$.	
	
%	iterative encode
%	
%	\begin{algorithm}[h!]
%	\caption{Iterative Encode}
%	\label{AE}
%	
%	\textbf{def $E(n,k,\{a_{v_1},a_{v_2},\ldots,a_{v_k}\})$:}
%	\begin{algorithmic}[1]
%	\STATE $e \gets 0$
%	\FOR {$i \gets 1..k$}
%		\STATE $e \gets e + \binom{n-v_i}{k-i+1}$
%	\ENDFOR
%	\RETURN $e$
%	\end{algorithmic}
%	
%	\end{algorithm}			
	
%	%%%%%%%%%%%%%%%%%%%%%%%%%%%%%%%%%%%%%%
%	\section{Decoding}
%	
%	In the lexicographical encoding, $\binom{n-i}{k-1}$ choices contain the $i$'th element as their first selection.
%	Since the choices are sorted in reverse order, this yields a simple algorithm to determine the first selection in an encoding $e$.
%	
%	\begin{algorithm}[h!]
%	\caption{Decode First Selection}
%	\label{ADFS}
%	
%	\textbf{def $decode\_first\_selection(e,n,k)$:}
%	\begin{algorithmic}[1]
%	\STATE $e' \gets e$
%	\STATE $i \gets n-k+1$
%	\WHILE{$\binom{n-i}{k-1} \le e'$}
%		\STATE $e' \gets e' - \binom{n-i}{k-1}$
%		\STATE $i \gets i - 1$
%	\ENDWHILE
%	\RETURN $i$
%	\end{algorithmic}
%	
%	\end{algorithm}
%	
%	For example, the choice $\{a_1,a_3\}$ from $\{a_1,a_2,a_3,a_4\}$ corresponds to the encoding $4$ (see figure \ref{Ex}).
%	Hence, on arguments $e=4$, $n=4$ and $k=2$, decode\_first\_selection should return $1$, as in ``$a_1$ is the first selection in the choice represented by $4$.''
%	
%	The invariant in decode\_first\_selection is that $e'$ is the relative encoding of all choices whose first element is at most $i$.
%	That is to say, $e' + \{\text{the lowest encoding that starts with $i$}\} = e$.
%	Hence, upon the algorithm's return, $e'$ is anywhere between $0$ and $\binom{n-i}{k-1}-1$.
%	
%	Combining the invariant with the fact that there are are exactly $\binom{n-i}{k-1}$ choices that start with the $i$'th element, we may recursively decode the rest of the list using $e'$, $n-i$ and $k-1$.
%	
%	\begin{algorithm}[h!]
%	\caption{Decode Delta}
%	\label{ADD}
%	
%	\textbf{def $decode\_delta(e,n,k)$:}
%	\begin{algorithmic}[1]
%	\IF {$k = 0$}
%	\RETURN $\left[ \right]$
%	\ELSE
%	\STATE $e' \gets e$
%	\STATE $i \gets n-k+1$
%	\WHILE{$\binom{n-i}{k-1} \le e'$}
%		\STATE $e' \gets e' - \binom{n-i}{k-1}$
%		\STATE $i \gets i - 1$
%	\ENDWHILE
%	\RETURN $i : decode\_delta(e',n-i,k-1)$
%	\ENDIF
%	\end{algorithmic}
%	
%	\end{algorithm}
%	
%	Algorithm \ref{ADD} has one small caveat: it produces the index differences, not the indices themselves.
%	That is to say, decode\_delta(5,4,2) would produce the list $\left[ 1,2 \right]$ instead of the expected list $\left[ 1,3 \right]$ (see figure \ref{Ex}).
%	This can be fixed in one linear pass.
%	
%	\begin{algorithm}[h!]
%	\caption{Decode}
%	\label{AD}
%	
%	\textbf{def $D(e,n,k)$:}
%	\begin{algorithmic}[1]
%	\STATE $deltas \gets decode\_delta(e,n,k)$
%	\STATE $sum \gets 0$
%	\STATE $indices \gets \left[ \right]$
%	\FOR {$delta \in deltas$}
%	\STATE $sum \gets sum + delta$
%	\STATE $indices . append(sum)$
%	\ENDFOR
%	\RETURN $indices$
%	\end{algorithmic}
%	
%	\end{algorithm}


%	\subsubsection{Top-Down Linear Search}
%	
%	Performing a linear search for $i$, starting from $i = n-k$, has one very large advantage:
%	the binomial coefficients may be incrementally computed in $O(1)$ time.
%	
%	For the first test, $f(i)=f(n-k)=\binom{n-(n-k)}{k}=1$.
%	Then, under the identity
%	
%	\begin{eqnarray}
%		\binom{n}{k} & = & \frac{n}{n-k+1}\cdot\binom{n-1}{k}
%		\label{I2}
%	\end{eqnarray}
%	we have $f(i+1)= \frac{n-i+1}{n-i-k}\cdot f(i)$.
%	
%	\subsubsection{Bottom-Up Linear Search}
%	
%	Like top-down linear search, bottom-up search may incrementally compute the binomial coefficient in $O(1)$ time.
%	However, unlike top-down search, computing the first coefficient takes $O(\min(k,n-k))$ time.
%	
%	Bottom-up search is better than top-down search in the expected number of iterations.
%	
%	\subsubsection{Binary Search}
%	
%	Binary search only takes $O(\log(n-k))$ iterations, but each iteration requires the computation of a binomial coefficient in $O(\min(k,n-k))$ time.
%	
%	\subsubsection{Lookup Table}
%	
%	The binomial coefficients could be stored in a lookup table.
%	This would make binary searching much more efficient.
%	
%	If space is not of concern, the a coefficient lookup table is actually not optimal - just have a lookup table for the whole encoding.
%	That said, a coefficient lookup table only needs $O(n^2)$ space, whereas a table for the entire encoding would need $\binom{n}{k}$ space.


%	%%%%%%%%%%%%%%%%%%%%%%%%%%%%
%	\subsection{Computing Binomial Coefficients}
%	
%	Algorithm \ref{ADD} uses a linear search to find the correct index.
%	The search is performed along a diagonal of Pascal's triangle, starting from $\binom{n-(n-k+1)}{k-1}=\binom{k-1}{k-1}$, and accumulating values as it heads down and left.
%	However, the partial sums of that diagonal can be found in the next diagonal, according to equation \ref{CE}. PICTURE REQUIRED
%	
%	Therefore we could ``improve'' algorithm \ref{ADD} by performing a binary search.
%	Except this improvement would not be faster - it precludes a more important optimization: the computation of the binomial coefficients.
%	
%	As presented, algorithm \ref{ADD} computes a binomial coefficient on each loop.
%	Computing the binomial coefficient on the $j$'th loop requires $j$ multiplications and divisions.
%	However, we may reuse the previous loop's computation, under the identity
%	
%	\begin{eqnarray}
%		\binom{n}{k} & = & \frac{n}{n-k+1}\cdot\binom{n-1}{k}
%		\label{I2}
%	\end{eqnarray}
%	
%	\begin{algorithm}[h!]
%	\caption{Decode Delta (Improved)}
%	\label{ADDI}
%	
%	\textbf{def $decode\_delta(e,n,k)$:}
%	\begin{algorithmic}[1]
%	\IF {$k = 0$}
%	\RETURN $\left[ \right]$
%	\ELSE
%	\STATE $e' \gets e$
%	\STATE $i \gets n-k+1$
%	\STATE $c \gets 1$
%	\WHILE{$c \le e'$}
%		\STATE $e' \gets e' - c$
%		\STATE $i \gets i - 1$
%		\STATE $c \gets \frac{n-i}{n-i-k+1}\cdot c$
%	\ENDWHILE
%	\RETURN $i : decode\_delta(e',n-i,k-1)$
%	\ENDIF
%	\end{algorithmic}
%	
%	\end{algorithm}


%	\vspace{0.2cm}
%	
%	Linear searches may compute $f(i)$ incrementally in $O(1)$ time, under the identity
%	
%	\begin{eqnarray}
%		\binom{n}{k} & = & \frac{n}{n-k+1}\cdot\binom{n-1}{k}
%		\label{I2}
%	\end{eqnarray}
%	
%	Linear searches, therefore, can be done in $O(n-k)$ time.
%	
%	Since choices with lower $i$ are more common, a search from $i=1$ up to $n-k$ would need fewer (expected) iterations than a search from $n-k$ down to $1$.
%	However, a bottom-up search needs to compute its first coefficient, $\binom{n-1}{k}$, in $O(\min(k,n-k))$ time; a top-down search's first coefficient, $\binom{n-(n-k)}{k}$, is just $1$.
%	
%	\vspace{0.2cm}
%	
%	Binary searches must compute each $f(i)$ in $O(\min(k,n-k))$ time, but only require $\log(n-k)$ iterations, for a total runtime of $O(\min(k,n-k)\cdot\log(n-k))$.
%	The benefits are especially apparent when $n \gg k$.
%	
%	For known $n$ and $k$, lookup tables may be used to speed up a binary search.
%	The table would need to contain $O(nk)$ elements.

%	\begin{lemma}
%		\label{kflip}
%		We may assume $k \le \frac{n}{2}$.
%	\end{lemma}
%	
%	\begin{proof}
%		A specific $k$-choice of $n$ elements can be represented by the $(n-k)$-choice of elements which were not selected, its dual.
%		We may convert a choice to its dual in $O(n)$ time.
%		
%		Both encode and decode take $\Omega(k)$ time, since they accept as input or produce as output a list of $k$ values.
%		
%		Therefore, when $k > \frac{n}{2}$, we may convert to/from the dual in $O(2k)$ time, which does not increase the $\Omega(k)$ runtime.
%		The dual will have $k \le \frac{n}{2}$.
%	\end{proof}

%	%%%%%%%%%%%%%%%%%%%%%%%%%%%%%%%%%%%%%%
%	\subsection{Decode Search}
%	
%	The decoding algorithm needs to find the minimum $i$ such that $\binom{n-i}{k} < e$.
%	Since $f(i) = \binom{n-i}{k}$ is a decreasing function, the correct $i$ may be found using a search algorithm.
%	
%	Binary searches require $O(\log(n-k))$ iterations.
%	On each iteration, $f(i)$ must be computed, which takes $O(\min(k,n-k))$ steps\cite{binom}.
%	Hence a binary search takes $O(\min(k,n-k)\cdot\log(n-k))$ time.
%	
%	Linear searches require $O(n-k)$ iterations.
%	However, linear searches may compute $f(i)$ incrementally in $O(1)$ time, under the identity
%	
%	\begin{eqnarray}
%		\binom{n}{k} & = & \frac{n}{n-k+1}\cdot\binom{n-1}{k}
%		\label{I2}
%	\end{eqnarray}
%	
%	This leaves linear searches at $O(n-k)$ time, which may or may not be better than binary search.
%
%	\vspace{0.2cm}
%	
%	What is the expected number of iterations of a bottom-up linear search?
%	We will assume that the encodings are uniformly distributed.
%	
%	There are $\binom{n-i}{k-1}$ encodings that start with the $i$'th element.
%	Since the linear search is bottom-up, a value of $i$ is determined to be correct on the $i$'th iteration.
%	Hence the expected number of iterations is
%	
%	\begin{eqnarray*}
%		E(I) & = & \frac{\sum i\cdot\{\text{\# of encodings that start with }i\}}{\text{\# of encodings}} \\
%		& = & \frac{1}{\binom{n}{k}} \sum_{i=1}^{n-k+1}{i\cdot\binom{n-i}{k-1}} \\
%		& = & \frac{1}{\binom{n}{k}} \sum_{r=1}^{n-k+1}{\sum_{i=r}^{n-k+1}{\binom{n-i}{k-1}}} \\
%		%& = & \frac{1}{\binom{n}{k}} \sum_{r=1}^{n-k+1}{\sum_{i=1}^{n-k+1-(r-1)}{\binom{n-(r-1)-i}{k-1}}} \\
%		& = & \frac{1}{\binom{n}{k}} \sum_{r=1}^{n-k+1}{\binom{n-r+1}{k}} \\
%		& = & \frac{1}{\binom{n}{k}} \binom{n+1}{k+1} \\
%		& = & \frac{n+1}{k+1} \\
%		\label{eq:qwsxsdgf}
%	\end{eqnarray*}
%	
%	In a top-down search, where the value of $i$ is determined to be correct on the $(n-k+2)-i$'th iteration, the expected number of iterations is $E(I) = (n-k+2)-\frac{n+1}{k+1}$.
%	
%	At a glance, it seems that bottom-up search is better than top-down search.
%	However, there is one other consideration.
%	In top-down search, where $i$ is initially $n-k+1$, the first $f(i)$ is $\binom{k-1}{k-1}$, which is simply $1$.
%	In bottom-up search, where $i$ is initially $1$, the first $f(i)$ is $\binom{n-1}{k-1}$, which must be computed in $O(\min(k,n-k))$ time.
%	This places the two approaches on comparable footing.


\end{document}
